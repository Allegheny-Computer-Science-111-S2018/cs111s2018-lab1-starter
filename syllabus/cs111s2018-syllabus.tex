%!TEX root=cs111S2018-syllabus.tex
% mainfile: cs111S2018-syllabus.tex 

%!TEX root=cs111s2018-syllabus.tex
% mainfile: cs111s2018-syllabus.tex 

% CS 111 style
% Typical usage (all UPPERCASE items are optional):
%       \input syllaabuspre
%       \begin{document}
%       \MYTITLE{Title of document, e.g., Lab 1\\Due ...}
%       \MYHEADERS{short title}{other running head, e.g., due date}
%       \PURPOSE{Description of purpose}
%       \SUMMARY{Very short overview of assignment}
%       \DETAILS{Detailed description}
%         \SUBHEAD{if needed} ...
%         \SUBHEAD{if needed} ...
%          ...
%       \HANDIN{What to hand in and how}
%       \begin{checklist}
%       \item ...
%       \end{checklist}
% There is no need to include a "\documentstyle."
% However, there should be an "\end{document}."
%
%===========================================================
\documentclass[11pt,twoside,titlepage]{article}
%%NEED TO ADD epsf!!
\usepackage{threeparttop}
\usepackage{graphicx}
\usepackage{latexsym}
\usepackage{color}
\usepackage{listings}
\usepackage{fancyvrb}
%\usepackage{pgf,pgfarrows,pgfnodes,pgfautomata,pgfheaps,pgfshade}
\usepackage{tikz}
\usepackage[normalem]{ulem}
\tikzset{
  %Define standard arrow tip
  %    >=stealth',
  %Define style for boxes
  oval/.style={
    rectangle,
    rounded corners,
    draw=black, very thick,
    text width=6.5em,
    minimum height=2em,
    text centered},
  % Define arrow style
  arr/.style={
    ->,
    thick,
    shorten <=2pt,
    shorten >=2pt,}
  }
\usepackage[noend]{algorithmic}
\usepackage[noend]{algorithm}
\newcommand{\bfor}{{\bf for\ }}
\newcommand{\bthen}{{\bf then\ }}
\newcommand{\bwhile}{{\bf while\ }}
\newcommand{\btrue}{{\bf true\ }}
\newcommand{\bfalse}{{\bf false\ }}
\newcommand{\bto}{{\bf to\ }}
\newcommand{\bdo}{{\bf do\ }}
\newcommand{\bif}{{\bf if\ }}
\newcommand{\belse}{{\bf else\ }}
\newcommand{\band}{{\bf and\ }}
\newcommand{\breturn}{{\bf return\ }}
\newcommand{\mod}{{\rm mod}}
\renewcommand{\algorithmiccomment}[1]{$\rhd$ #1}
\newenvironment{checklist}{\par\noindent\hspace{-.25in}{\bf Checklist:}\renewcommand{\labelitemi}{$\Box$}%
\begin{itemize}}{\end{itemize}}
  \pagestyle{threepartheadings}
  \usepackage{url}
  \usepackage{wrapfig}
  % removing the standard hyperref to avoid the horrible boxes
  %\usepackage{hyperref}
  \usepackage[hidelinks]{hyperref}
  % added in the dtklogos for the bibtex formatting
  %\usepackage{dtklogos}
  %=========================
  % One-inch margins everywhere
  %=========================
  \setlength{\topmargin}{0in}
  \setlength{\textheight}{8.5in}
  \setlength{\oddsidemargin}{0in}
  \setlength{\evensidemargin}{0in}
  \setlength{\textwidth}{6.5in}
  %===============================
  %===============================
  % Macro for document title:
  %===============================
  \newcommand{\MYTITLE}[1]%
  {\begin{center}
      \begin{center}
        \bf
        CMPSC 111\\Introduction to Computer Science I\\
        Spring 2018
        \medskip
      \end{center}
      \bf
      #1
    \end{center}
    }
  %================================
  % Macro for headings:
  %================================
  \newcommand{\MYHEADERS}[2]%
  {\lhead{#1}
    \rhead{#2}
    %\immediate\write16{}
    %\immediate\write16{DATE OF HANDOUT?}
    %\read16 to \dateofhandout
    \def \dateofhandout {January 17, 2018}
    \lfoot{\sc Handed out on \dateofhandout}
    %\immediate\write16{}
    %\immediate\write16{HANDOUT NUMBER?}
    %\read16 to\handoutnum
    \def \handoutnum {1}
    \rfoot{Handout \handoutnum}
    }

  %================================
  % Macro for bold italic:
  %================================
  \newcommand{\bit}[1]{{\textit{\textbf{#1}}}}

  %=========================
  % Non-zero paragraph skips.
  %=========================
  \setlength{\parskip}{1ex}

  %=========================
  % Create various environments:
  %=========================
  \newcommand{\PURPOSE}{\par\noindent\hspace{-.25in}{\bf Purpose:\ }}
  \newcommand{\SUMMARY}{\par\noindent\hspace{-.25in}{\bf Summary:\ }}
  \newcommand{\DETAILS}{\par\noindent\hspace{-.25in}{\bf Details:\ }}
  \newcommand{\HANDIN}{\par\noindent\hspace{-.25in}{\bf Hand in:\ }}
  \newcommand{\SUBHEAD}[1]{\bigskip\par\noindent\hspace{-.1in}{\sc #1}\\}
  %\newenvironment{CHECKLIST}{\begin{itemize}}{\end{itemize}}


\usepackage[compact]{titlesec}

\begin{document}
\MYTITLE{Syllabus}
\MYHEADERS{Syllabus}{}

\subsection*{Course Instructor}
Dr.\ Janyl\ Jumadinova\\
\noindent Office Location: Alden Hall 105 \\
\noindent Office Phone: +1 814-332-2881 \\
\noindent Email: \url{jjumadinova@allegheny.edu} \\
\noindent Web Site: \url{http://www.cs.allegheny.edu/sites/jjumadinova/} 

\subsection*{Instructor's Office Hours}

\begin{itemize}
  \itemsep 0em
  \item Monday and Friday: 2:00 pm -- 3:00 pm (10 minute time slots)
  \item Tuesday and Thursday: 10:30 am -- 12:30 pm (15 minute time slots)
\end{itemize}

\noindent To schedule a meeting with me during the office hours, please go to  \url{http://cs.allegheny.edu/sites/jjumadinova/schedule}, click on the ``book an appointment'' link 
 and select the available date and time of your choice. You can schedule an appointment outside of my office hours via course's Slack channel or email.

\subsection*{Course Meeting Schedule}

\noindent \textbf{Lecture, Discussion, and Group Work session}: Alden 101, \\ Section 1 - Monday and Wednesday 9:00am to 9:50am, Section 2 - Monday and Wednesday 11:00am to 11:50am \\
\textbf{Laboratory session}: Alden 101, \\ Section 1 - Thursday 2:30pm to 4:20pm, Section 2 - Wednesday 2:30pm to 4:20pm \\
\textbf{Practical session}: Alden 101, Section 1 - Friday 9:00am to 9:50am, Section 2 - Friday 11:00am to 11:50am

\subsection*{Course Resources}

\noindent \textbf{Course Web page}: \url{http://cs.allegheny.edu/sites/jjumadinova/teaching/111} \\
\noindent You can find the most up-to-date schedule of the course and the required readings on the course's page.

\noindent \textbf{Department Website}: \url{http://www.cs.allegheny.edu/teaching/teachingassistants/} \\
\noindent You can view the office hours of teaching assistants and tutors on the department's website.

\noindent \textbf{Sakai page}: \url{https://sakai.allegheny.edu/} \\
\noindent The course page on Sakai will only be used for reporting student grades and online quizzes and exams.

\noindent \textbf{Github}: \url{https://github.com/} \\
Github, a cloud based system, will be used for sharing course materials by the instructor and for submitting assignments by the students. 

\noindent \textbf{Slack channel}: \url{https://cs111s2018.slack.com/} \\
Slack will be used for discussion and communication outside of the classroom. 

\subsection*{Academic Bulletin Course Description}

\begin{quote}

An introduction to the principles of computer science with an emphasis on creative expression through the medium of a programming language. Participating in hands-on activities that often require teamwork, students learn the computational structures needed to solve problems and produce artifacts such as interactive games and computer-mediated art and music. Students also learn how to organize and document a programs source code so that it effectively communicates with the intended users and maintainers. Additionally, the introduction includes an overview of the discipline of computer science and computational thinking. During a weekly laboratory session students use state-of-the-art technology to complete projects, reporting on their results through both written reports and oral presentations. \\
\noindent Prerequisite: Knowledge of elementary algebra. \\
\noindent Distribution Requirements: ME, SP. 


\end{quote}

\noindent The course follows three parallel tracks. In the lectures we will learn and practice basic computer science fundamentals, in the practical sessions you will reinforce that knowledge with short practical exercises, while in the laboratory sessions you will have a larger hands-on experience with problem solving and writing programs. The laboratory and practical sessions will be usually tied to the lectures.

\subsection*{Course Objectives}

The process of implementing and evaluating correct and efficient software involves the application of many interesting
theories, techniques, and tools.  In addition to learning problem solving and computational thinking skills, this class
will teach students how to use, design, implement, and test software developed in an object-oriented programming language.
Students will learn more about fundamental concepts such as data types, conditional logic, and iteration while also
discovering how to use single-dimensional arrays and lists. Students will be provided with opportunities to implement graphical and musical
applications.  Students also will gain hands-on experience in the use, design, implementation, and testing of software
during the laboratory and practical sessions and a final project.  Along with learning more about how to effectively
work in a team of diverse software developers, students will enhance their ability to write and present ideas about software in
a clear, concise, and compelling fashion.  Students will also develop an understanding of the fascinating connections
between computer science and other disciplines in the social and natural sciences and the humanities.

\subsection*{Learning Objectives}

At the completion of this semester, students must have a strong grasp of the basics of the object-oriented programming
paradigm and an introductory knowledge of topics like conditional logic, iteration, exceptions, and
applied areas of computer science.  Also, students should be able to handle many of the important, yet accidental, aspects of
implementing programs in Java.  
Students should have a toolkit of programming language constructs that they can use to respond to the challenges that
they encounter during the development and evaluation of software. Finally, students should demonstrate the ability to
use both in-person discussions and cutting-edge software tools to effectively communicate and collaborate with a group
of diverse team members.

\subsection*{Required Textbooks}

{\em Java Software Solutions: Foundations of Program Design}, $8^{th}$ edition, by John Lewis and William Loftus.

\noindent Along with the required books and handouts, you will be assigned to read additional articles from a wide variety of
conference proceedings, journals, and the popular press.

\subsection*{Class Policies}

\subsubsection*{Grading}
The grade that a student receives in this class will be based on the following categories. 
All percentages are approximate and it is possible for the assigned
percentages to be changed during the academic semester, if a need arises to do so.

\begin{center}
\begin{tabular}{|c|c|}
\hline
Class Participation & $15\%$\\
\hline
Laboratory Assignments  & $30\%$\\
\hline
Practical Assignments  & $10\%$\\
\hline
Midterm Exam & $15\%$\\
\hline
Final Exam & $15\%$\\
\hline
Final Project & $15\%$\\
\hline
\hline
\textbf{Total} & $100\%$\\
\hline
\end{tabular}
\end{center}

\noindent
These grading categories have the following definitions:
\vspace*{-.1in}

\begin{itemize}

  \item {\em Class Participation}: All students are required to actively participate
    during all of the class sessions. Your participation will take forms of completing class exercises, answering questions about the required reading assignments, and asking constructive questions. You must also
    regularly participate in the discussions on the Slack channels for this course.
%\vspace*{-.05in}
  \item {\em Laboratory Assignments}: Lab assignments invite students to explore different techniques for designing,
    implementing, evaluating, and documenting software solutions to challenging problems that often have a connection to
    real-world concerns.  To best ensure that students are ready to develop software in both other classes at
    Allegheny College and after graduation, students will complete assignments both on an individual basis and in teams.
%\vspace*{-.05in}   
  \item {\em Practical Assignments}: Practical assignments are intended to give students a practice using the new concepts without being afraid to fail. These are short assignments to be completed by the end of the class period, and are graded only as `attempted' or `not attempted'. 
%\vspace*{-.05in}    
  \item {\em Midterm Examination}: The midterm is an hour-long cumulative test covering all of the material from the
    class, practical, and laboratory sessions. Unless prior arrangements are made with the course instructor, all
    students will be expected to take this test on the scheduled date. The finalized date for the midterm will be announced at least one week in advance of the scheduled date, tentatively it will be held a few days before the spring break. 
%\vspace*{-.05in}
  \item {\em Final Examination}: The final examination is a cumulative test.  By enrolling in this
    course, students agree that, unless there are extenuating circumstances, they will take the final examination
   during college's scheduled date and time.
%\vspace*{-.05in}    
  \item {\em Final Project}: This project will present you with an opportunity to design
    and implement a correct and carefully evaluated solution for a particular problem. Completion of the final project will require you to apply all of the knowledge and skills that you have acquired during the course of the semester to solve a problem. The details for the final project will be given approximately one month before the finals week.
\end{itemize}


\subsubsection*{Assignment Submission}

All assignments will have a stated due date and are to be turned in electronically on that due date; all assignments must have headers with your name, date and the Honor Code pledge of the
student(s) completing the work.  You must follow proper procedures for submitting your completed programs in order for them to be graded. You will be given instructions on how to do that with your first programming assignment. \\

\noindent Late assignments will be accepted for up to one week past the assigned due date with a
15\% penalty. All of the late assignments must be submitted at the beginning of the session that is scheduled one week
after the due date. Unless special arrangements are made with the course instructor, no assignments will be accepted
after the late deadline. For any assignment completed in a group, students must also turn in a one-page retrospective that
describes each group member's contribution to the submitted deliverables and the dynamics of their team work. 

\subsubsection*{Attendance}

It is mandatory for all students to attend all of the class, practical, and laboratory sessions. If you will not be able
to attend a session, then please see me at least one week in advance to describe your situation. In case you missed a class because of an emergency, please notify me as soon as possible. 
Students who miss more than five unexcused sessions will have their final grade in the course reduced by one letter grade. 
 Frequent or prolonged absences due to illness should be documented by the student's doctor, the Health Center, the Dean of Students' Office, or the office of Student Disability Services. If you need to miss class due to a religious observance, please speak to me in advance to make appropriate arrangements. 

\subsubsection*{Use of Laboratory Facilities}

Throughout the semester, we will investigate many different software tools that computer scientists use during the
design, implementation, and evaluation of algorithms.  The course instructor and the department's systems administrator
have invested a considerable amount of time to ensure that our laboratories support the completion of all of the
assignments and projects.  To this end, students are required to complete all of the laboratory and practical
assignments and the final project while using the department's laboratory facilities. The course instructor and the
systems administrator normally do not assist students in configuring their personal computers.

\noindent You may access your computer science account remotely by following instructions on:\\
{\url{http://www.cs.allegheny.edu/about/x2go/}}

\subsubsection*{Class Preparation}

In order to minimize confusion and maximize learning, students must invest time to prepare for the class discussions,
lectures, and practical and laboratory sessions. During the class periods, the course instructor will often pose  questions that could require group discussion, the creation of a program or data set, a vote on a thought-provoking issue, or a group presentation.  
In order to help students remain organized and effectively prepare for classes, the course instructor will maintain a class schedule with reading assignments and presentation slides. 

\subsection*{Seeking Assistance}

Students who are struggling to understand the knowledge and skills developed in a class or a laboratory
session are encourage to seek assistance from the course instructor. Students who need the course instructor's assistance should schedule a meeting through her Web site. \\

\noindent \emph{A Note on extenuating circumstances}

\noindent If you should find yourself in difficult circumstances that significantly interfere with your ability to prepare for this class and to complete assignments, please inform me immediately so that we can work something out together! Do not wait until the last day of class to ask for exceptions to what is stated in this syllabus. In such a situation, you may also find it helpful to contact one of the available resources on campus: \\

\noindent The Maytum Learning Commons, Library/Academic Commons, 814-332-2898 \\
\noindent You may request an individual tutor through Learning Commons:\\ {\url{http://sites.allegheny.edu/learningcommons/tutoring/}}\\

\noindent Allegheny	College	Counseling	Center,	Reis 	Hall,	 814	-332-4368 \\

\noindent Winslow	Health	Center,	Schultz	Hall	, 814-332-4355 \\

\noindent Allegheny	College	Chaplain,	Reis	 Hall,	 814-332-2800

\subsection*{Special Needs and Disability}
Students with disabilities who believe they may need accommodations in this class are encouraged to contact Disability Services at (814) 332-2898. Disability Services is part of the Learning Commons and is located in Pelletier Library. Please do this as soon as possible to ensure that approved accommodations are implemented in a timely fashion.

\subsection*{Honor Code}
All students enrolled at Allegheny College are bound by the Honor Code. It is expected that
your behaviour will reflect that commitment. To this end, we expect that you will adhere to the 
following Department Policy:

\begin{center} \textbf{ Department of Computer Science Honor Code Policy } \end{center}
%\vspace*{-.05in}
It is recognized that an important part of the learning process in any course, and particularly in computer science, derives from thoughtful discussions with teachers, student
assistants, and fellow students. Such dialogue is encouraged. However, it is necessary
to distinguish carefully between the student who discusses the principles underlying a
problem with others, and the student who produces assignments that are identical to,
or merely variations on, someone else's work. It will therefore be understood that all
assignments submitted to faculty of the Department of Computer Science are to be
the original work of the student submitting the assignment, and should be signed in
accordance with the provisions of the Honor Code.  Appropriate action will be taken when assignments give evidence that they were derived from the work of others.\\

\noindent You are encouraged to periodically review the specifics of the Honor Code as stated in the
College Catalogue and the Compass. 

\end{document}
